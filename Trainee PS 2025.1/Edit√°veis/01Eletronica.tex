\section{Eletrônica}

A área de eletrônica na equipe é fundamental. Somos responsáveis por entender, selecionar e conectar os componentes eletrônicos de forma segura e eficiente. Isso inclui o estudo de controladoras de voo, ESCs, sensores, baterias, entre outros, garantindo que todos os sistemas estejam funcionando em conjunto.

Além da integração dos sistemas, a eletrônica também é essencial para a implementação de novas ideias, como a automação de decisões do drone com o uso de sensores adicionais. Nossa atuação também abrange a manutenção preventiva e corretiva dos componentes eletrônicos, bem como projetos paralelos que, embora não estejam diretamente ligados ao drone, contribuem de forma indireta para sua operação.

\subsection*{Objetivo}

Estudar, compreender e representar o sistema eletrônico completo de um drone. A proposta envolve pesquisar os principais componentes, entender seu funcionamento e montar um esquema de conexão com base nesses estudos. A montagem e a escolha dos componentes devem considerar não apenas a capacidade do drone de voar, mas também sua eficácia na execução da missão proposta no início deste relatório.

\subsection*{Etapas do Projeto}

Para a organização e melhor entendimento do projeto, as seguintes etapas devem ser seguidas:

\subsubsection*{Etapa 1 – Levantamento dos Componentes Principais}

Nesta etapa inicial, o foco é projetar um drone funcional. Os candidatos devem pesquisar todos os principais componentes necessários para o funcionamento do sistema, considerando suas conexões, alimentação e demais requisitos técnicos.

\textbf{Componentes obrigatórios:}
\begin{itemize}
    \item Controladora de voo (Flight Controller)
    \item ESCs (Electronic Speed Controllers)
    \item Motores (Brushless DC)
    \item Bateria (LiPo)
    \item Hélices
    \item GPS
    \item Sistema de rádio (receptor/transmissor)
    \item Módulo de telemetria
    \item Um sensor adicional que auxilie na missão
\end{itemize}

\subsubsection*{Etapa 2 – Pesquisa Detalhada de Cada Componente}

Entender o funcionamento de cada componente eletrônico é essencial. Conhecer suas características técnicas e funções auxilia nas decisões do projeto. Para cada componente escolhido, devem ser levantadas as seguintes informações:

\begin{itemize}
    \item Nome completo e modelo (quando aplicável)
    \item Função no sistema
    \item Tensão e corrente de operação
    \item Conexões (pinos, interfaces: PWM, UART, I2C etc.)
    \item Comunicação com a controladora de voo
    \item Dados relevantes do \textit{datasheet}
    \item Requisitos de alimentação
    \item Cuidados de operação ou limitações
\end{itemize}

\subsubsection*{Etapa 3 – Elaboração do Diagrama de Conexões}

Com os componentes definidos e compreendidos, deve-se elaborar um diagrama representando como todos os módulos se conectam eletricamente. Isso inclui fios, sinais transmitidos e origens da alimentação.

\textbf{Recomendações:}
\begin{itemize}
    \item Utilizar softwares como Fritzing, KiCad, Lucidchart, Draw.io, EasyEDA etc.
    \item Identificar claramente as linhas de alimentação, sinal e terra (GND)
    \item Adicionar legendas e setas indicando a direção dos sinais e da energia
\end{itemize}

\subsubsection*{Etapa 4 – Relatório Final}

Durante o desenvolvimento, é essencial registrar cada etapa de forma clara e objetiva. O relatório final deve compilar todas as etapas anteriores.

\textbf{O que documentar:}
\begin{itemize}
    \item Lista dos componentes pesquisados: nome, função, tensão de operação, corrente típica e interfaces
    \item Justificativa de escolha: motivos da seleção de cada componente (custo, compatibilidade, desempenho), acompanhada de uma tabela de valores
    \item Diagrama esquemático das conexões: pode ser digital ou feito à mão, desde que legível e tecnicamente correto
    \item Anotações técnicas: tensões, pinos usados, tipos de sinais (digital, analógico, PWM, UART etc.)
    \item Dificuldades e soluções: problemas enfrentados na pesquisa ou montagem e como foram resolvidos
\end{itemize}



