\section{Introdução}

A elaboração deste projeto tem como objetivo proporcionar um aprendizado abrangente sobre diversos temas relacionados ao campo de processamento de imagens, utilização de drones multicópteros e tecnologias associadas através do envolvimento dos participantes com assuntos e situações reais da equipe. O  documento será elaborado ao longo de nove semanas, a partir da data de 15/05/2025 e entregue até 21/07/2025. Após essa etapa, ocorrerá uma apresentado deste documento para uma banca avaliadora. O resultado será divulgado até dia 23/07/2025 e a efetivação acontecerá na Reunião Geral, juntamente com a equipe, no dia 26/07/2025.\\

Dessa forma, deverá ser produzido um projeto teórico validado real de um quadrirrotor. Ao final do projeto, um relatório técnico geral deverá ser apresentado, com até 50 páginas, detalhando as ideias, mecanismos, lógicas, estratégias, componentes, dentre outras contribuições que os candidatos considerarem importante de serem abordados em cada área. 

Os líderes das áreas, apresentados no grupo geral do processo no Whastapp estarão disponíveis e dispostos a ajudar e colaborar com dúvidas que os candidatos possam ter durante a elaboração do projeto, porém os líderes tem autonomia de não responderem determinadas perguntas que não acharem pertinente de resposta. Ademais, cada área deve ter \textbf{OBRIGATORIAMENTE} um líder de área, e o projeto deve ter \textbf{OBRIGATORIAMENTE} um gerente de projeto, no qual será decidido pelos candidatos. Encorajamos a comunicação e a troca de conhecimentos para garantir a qualidade e a precisão das análises realizadas. Ademais, visando desfrutar de uma bom aprendizado, recomendamos um ponderamento na utilização de inteligencias artificiais, visto que caso seja perceptível que o candidato utilizou de forma demasiada tal ferramenta e não consiga explicar seus mecanismos e funcionalidades, o candidato estará sujeito a expulsão do processo seletivo. \\

A avaliação acontecerá a partir da escrita do relatório, da apresentação do projeto e das avaliações feitas pelos líderes de área sobre os candidatos ao longo de todo o processo, ficando a cargo do líderes de cada área e aos capitães decidirem o resultado de cada candidato.

